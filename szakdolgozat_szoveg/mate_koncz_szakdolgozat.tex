\documentclass[a4paper,12pt]{report}
\usepackage[T1]{fontenc}
\usepackage[utf8]{inputenc}
\def\magyarOptions{defaults=hu-min}
\usepackage[magyar]{babel}
\usepackage{amsthm, amssymb,amsmath,hyperref}
\usepackage{enumerate, graphicx, xcolor}

\usepackage{chngcntr}
\counterwithout{figure}{chapter}
\counterwithout{figure}{section}
\counterwithout{figure}{subsection}
%\usepackage{pgf,tikz,float}
%\usepackage{tikzlings}
%\usepackage{tikzducks}
%\usetikzlibrary{arrows}
\usepackage[nobysame]{amsrefs}
%\usepackage{amsmath}

\usepackage{geometry}
 \geometry{
 a4paper,
 total={160mm,247mm},
 left=25mm,
 top=25mm,
 }


\newtheorem{theo}{tétel}[section]
\newtheorem{defin}[theo]{definíció}
\newtheorem{lemma}[theo]{lemma}
\newtheorem{all}[theo]{állítás}
\newtheorem{kov}[theo]{következmény}

\theoremstyle{definition}
\newtheorem{definition}[theo]{definíció}

\theoremstyle{remark}
\newtheorem{megj}[theo]{megjegyzés}




\date{today}

\linespread{1.3}

\usepackage{listings}
\usepackage{color}

\definecolor{dkgreen}{rgb}{0,0.6,0}
\definecolor{gray}{rgb}{0.5,0.5,0.5}
\definecolor{mauve}{rgb}{0.58,0,0.82}

\lstset{frame=tb,
  language=Java,
  aboveskip=3mm,
  belowskip=3mm,
  showstringspaces=false,
  columns=flexible,
  basicstyle={\small\ttfamily},
  numbers=none,
  numberstyle=\tiny\color{gray},
  keywordstyle=\color{blue},
  commentstyle=\color{dkgreen},
  stringstyle=\color{mauve},
  breaklines=true,
  breakatwhitespace=true,
  tabsize=3
}



\begin{document}


\pagenumbering{roman}

%Elso  oldal 
\thispagestyle{empty}

\begin{center}
\vspace*{0.2cm} {\Large\bf Szegedi Tudományegyetem}
\vspace{0.3cm}

{\Large\bf Természettudományi és Informatikai Kar}
\vspace{0.3cm}

{\Large\bf Informatikai Intézet, Szoftverfejlesztés Tanszék}
\vspace{3cm}



{\Large SZAKDOLGOZAT}

\vspace*{1.5cm}

{\LARGE\bf Eseményszervező webalkalmazás konfigurációs lehetőségekkel}

Configurable event organizing web application



\vspace*{4cm}

{\large
\begin{tabular}{c@{\hspace{2cm}}c}
\emph{Készítette:}     &\emph{Témavezető:}\\
\bf{Koncz Máté}  &\bf{Dr. Pengő Edit}\\
Programterveő Informatikus BSc hallgató    & egyetemi docens\\
&
\end{tabular}
}

\vspace*{1,5cm}

{\Large Szeged\\ \vspace{2mm} 2025}
\end{center}

%masodik oldal osszefogalalo
\begin{abstract}
A dolgozat tartalmának rövid (max. 1 oldal) összefoglalása. A következő részekből áll: rövid irodalmi összefoglaló, a dolgozat elkészítéséhez használt módszerek, eredmények, konklúzió

{\bf Kulcsszavak:} a dolgozat tartalmára specifikusan jellemző 4-6 szó, egymástól vesszővel elválasztva
\end{abstract}



\newpage


\pagebreak

\tableofcontents
\pagebreak
%\listoffigures
%\pagebreak








\chapter{Bevezetés}
\pagenumbering{arabic}

\section{Motiváció}

Az alkalmazás ötlete még gimnazista koromban született meg. Az osztálytársaimmal gyakran jártunk össze focizni, ám ezeket az alkalmakat mindig komoly kihívás volt megszervezni. Az legtöbb alkalom azért maradt el, mert nem sikerült olyan időpontot találni, amelyre legalább 8 ember biztosan el tudott jönni. A szavazások nehezen folytak le, rendszerint a messenger-en küldött 'foci péntek délután?', 'foci szombat délelőtt?' stb. üzenetekre adott reakciók száma alapján dőlt el a focizás időpontja. Az alkalmazásom az ehhez hasonló események szervezését könnyíti meg, ahol a legnagyobb kihívás az ideális időpont megtalálása, hiszen a legfontosabb, hogy az minél több emnbernek feleljen meg.

\chapter{Fő funkciók bemutatása}

\section{A felhasználók kezelése}

Az alkalmazásban a felhasználók kezelése nem a legfontosabb feladat, számos, hasonló alkalmazásban gyakori funkciókat, mint a felhasználói profil szerkesztése, vagy a profil törlése, nem valósítottam meg. A felhasználók autentikációja és autorizációja azonban fontos feladat.

	\subsection{Regisztráció}

A felhasználó regisztrálhat egy egyedi felhasználónévvel, egyedi email-címmel, és a legalább 8 karakter hosszú jelszavának kétszer történő megadásával. Amennyiben a megadott névvel vagy email-címmel már foglalt, a rendszer értesíti erről a felhasználót.

	\subsection{Bejelentkezés, kijelentkezés}

A helyes felhasználónév és jelszó megadásával a létrejön egy session, ami 4 órán keresztül biztosít hozzáférést az alkalmazáshoz a felhasználó számára. A session lejárta után az alkalmazás automatikusan kijelentkezteti a felhasználót.

	\subsection{Autorizáció}

Az események adataihoz csak a szervező és a meghívottak férhetnek hozzá. Az eseményeket és az ezekhez tartozó naptárakat csak a szervezők szerkeszthetik.

\section{Események kezelése}

	\subsection{Létrehozás}

Az eseményeket a nevük és egy rövid leírás megadásával lehet létrehozni. Naptár hozzáadása opcionális, ez később is megadható, a részleteiről részletesebben lejjeb írtam.  Az esemény létrehozásakor a szervező automatikusan a meghívottak közé is bekerül.

	\subsection{Csatlakozás}

A szervező és a meghívottak számára látható az esemény meghívó-kódja. Ők ezt a kódot továbbíthatják másoknak, akik ezt a kódot használva a meghívottak közé kerülnek.

	\subsection{Extra mezők}

A szervező extra mezőket adhat az eseményhez. Ezekhez egy cím és a mező lehetséges értékei tartoznak. A mező aktuális értékét beállíthatja a szervező, vagy szavazásra is bocsájthatja azt. Emellett a szervező azt is beállíthatja, hogy a meghívottak vehetnek-e fel új lehetséges értékeket az adott mezőhöz.

	\subsection{Eseménysémák használata}

Egy esemény résztvevője jogosult arra, hogy az eseményből eseménysémát készítsen. A séma tartalmazza az eseményhez rendelt extra mezőket azok tulajdonságaival és lehetséges értékeivel együtt, és alkalmas arra, hogy a használatával új eseményt lehessen létrehozni, amely extra mezői ugyanezekkel a tulajdonságokkal rendelkeznek.

	\subsection{Esemény adatainak véglegesítése}

Egy esemény szervezője véglegesítheti egy esemény adatait (ebbe az esemény időpontja és az extra mezők értékei tartoznak bele).  Ezután az esemény adatai nem módosíthatóak, a meghívottak pedig értesítést kapnak az esemény véglegesítéséről. A művelet visszavonható, ekkor újra módosíthatóvá válnak az adatok, a meghívottak pedig erről is értesítést kapnak.

\section{Naptárak kezelése}

A naptárfelület és a hozzá tartozó funkciók az alkalmazás legfontosabb részét alkotják, hiszen ennek a feladata, hogy segíse az esemény szervezőjét az ideális időpont kiválasztásában. (((ábra a naptárfelületről)))

	\subsection{Létrehozás}

Naptár hozzáadása az időintervallum (legfeljebb 60 nap) és az időzóna megadásával történik. Az intervallum kezdő időpontja nem lehet a múltban, és az intervallum kezdete és vége nem lehet ugyanaz az időpont.

	\subsection{Szerkesztés}

Az esemény szervezője szűkítheti a választható időpontok körét. Az órákat és napokat letilthatja egyenként és naponta, illetve hetente ismétlődő jelleggel is. A szervező ezeket a műveleteket bármikor visszavonhatja.

	\subsection{Vélemény megadása}

Az események résztvevői (ebbe az esemény szervezője is beletartozik) minden elérhető időpontról véleményt formálhatnak. Egy időpontról megadható, hogy megfelelő, elfogadható, vagy nem megfelelő. Vélemény megadásához órákat vagy egész napokat lehet kijelölni. A vélemény rögzítése visszavonható művelet.

	\subsection{Statisztikák megtekintése}

Az események résztvevői minden időpont esetében megtekinthetik, hogy azokhoz mely felhasználók milyen véleményt rögzítettek. Az esemény szervezője átfogóbb statisztikákat is lekérhet. (((ide ábrák kellenek)))

	\subsection{Időzónák kezelése}

Az időzóna megadása azért fontos, mert a  kiválasztható időpontok egész órák, amikből az óraátállítások esetén nem 24 van. A óraátállítások pedig különbözhetnek időzónánként. Fontos, hogy a felhasználók ne tudjanak olyan órákat kijelölni, amik nem léteznek, hiszen ez megtévesztő, és az esetleges plusz órát is jelölni kell.

\chapter{Felhasznált technológiák}

\section{Spring Boot}

A Spring Framework egy Java üzleti alkalmazások fejlesztését segítő keretrendszer. Legfontosabb funkciói közé tartozik a függőség befecskendezés és a Model-View-Controller modell (MVC) támogatása.

A backend alkalmazásomat a Spring Boot\cite{Springwebsite} használatával készítettem el. Ennek segítségével gyorsabban és egyszerűbben tudom elkezdeni a Spring Frameworkre épülő alkalmazásom fejlesztését, mivel a Spring Boot projektek rendelkeznek alapértelmezett konfigurációs beállításokkal, nekem ezeket csak akkor kell átírnom, ha valahol az alapértelmezettől eltérő viselkedést akarok elérni. A fejlesztést InteliJ IDEA (Community Edition)\cite{IDEAwebsite} környezetben végeztem.

	\subsection{Jakarta Persistence API Hibernate ORM-el}

A Hibernate\cite{Hibernatewebsite} segítségével végeztem objektum-relációs leképezéseket a backend alkalmazásomban, ezt azonban a Java Persistence API-n (JPA)\cite{JPAwebsite} keresztül használtam. A JPA egy interfészt szolgáltat, amely segítségével a Java kódban kezelhetem a relációs adatbázisomban (esetemben a MariaDB szerveremen) tárolt adatokat. A JPA-val emellett saját natív query-ket is írtam, ahol az adatokat egyéni módon akartam kezelni.

	\subsection{H2 In-memory DB}

A H2\cite{H2website} memórián belüli adatbázis megvalósítását az integrációs tesztjeimhez használtam. Mivel ez az adatbázis minden futtatás esetében újra inicializálódik, fejlesztés közben egyszerűbb a tesztjeimet futtatnom, hiszen nem kell minden, a leképezendő objektumok kapcsolatait érintő változás esetén a megfelelő strukturális változtatásokat véghezvinnem az adatbázisban. Emellett általánosságban egyszerűbben kezelhető, hiszen nincs szükségem adatbázisszerverre.

\section{MariaDB}

A backend alkalmazás éles futtatása során egy MariaDB\cite{Mariawebsite} adatbázisszerverre kapcsolódik. A MariaDB egy nyílt forráskódú, SQL alapú relációs adatbázis-kezelő rendszer (RDBMS). Más hasonló adatbázisszerverek közül azért erre esett a választásom, mert a tanulmányaim során ezt használtam a legtöbbet.

\section{Angular}

Az Angular\cite{Angularwebsite} egy Google által fejlesztett, nyílt forráskódú, TypeScript alapú keretrendszer webalkalmazások fejlesztésére. Az applikációm frontend részét ezzel valósítottam meg. A választásom azért erre esett, mert a hozzá hasonló keretrenszerek közül ezt ismerem a legjobban, és úgy ítéltem meg, hogy alkalmas modern, reszponzív felhasználói felületek létrehozására. A fejlesztést Visual Studio Code\cite{VSCwebsite} környezetben végeztem.

	\subsection{Angular Material}

A frontenden nagy számban használtam az Angular Material\cite{Materialwebsite} könyvtár komponenseit. Az Angular Material a Google hivatalos UI komponenskönyvtára, számos termékükben használják.

	\subsection{RxJs}

Az RxJs\cite{Rxjswebsite} könyvtár aszinkron műveletek kezelését segítő függvényeket és típusokat tartalmaz. A frontendemen könyvtárban található Observable típust használtam az aszinkron API lekérdezések megvalósításához, valamint a Subject típust használtam azokban az esetekben, amikor gyerek komponensben kellett szülő komponensben bekövetkező eseményre reagálni.

\section{git}

A git\cite{Gitwebsite} verziókövető rendszert használtam a teljes fejlesztési folyamat során. Hogy a konzulensem is követhesse a projekt alakulását, létrehoztam egy nyilvános Github\cite{GitHubwebsite} repository-t, és oda sűrűn pusholtam a local repository commit-jait.


\chapter{Az API fejlesztése}

\section{A model réteg}

A model osztályaim egyszerű POJO-k (Plain Old Java Object). Az osztályokban sűrűn használtam a JPA objektum relációs leképzésekhez szükséges annotációit valamint az általam szerializáláshoz használt Jackson ObjectMapper működését segítő annotációkat. Az alkalmazás további fejlesztése során gyakran használtam a Set interfészt, ezért a model osztályokban felülírtam az equals() és hashCode() metódusokat. Ezek mellett copy konstruktorokat is létrehoztam, hogy lehetőségem legyen az osztályok példányainak egyszerű másolására.

\section{A repository réteg}

Minden model osztályhoz létrehoztam egy repository interfészt, amelyeket a JPARepository interfésztől származtattam. A JPARepositoryban már az alapvető CRUD műveletek meg vannak valósítva, én azonban néhány helyen kiegészítettem ezeket új műveletekkel (például az eseményekre nem csupán az ID-jük, hanem a hozzájuk rendelt meghívó kód alapján is lehet keresni az adatbázisban).

\section{A service réteg}

A service komponenseket nem az általuk használt repository-k, hanem a funkcióik szerint különítettem el, így egy service több repository-val is rendelkezhet. A service-ek között tartalmazó kapcsolat is előfordul, pédául a naptárakkal kapcsolatos műveleteket végző CalendarService metódusai kizárólag az eseményeket kezelő EventService-ből hívódnak meg. A service osztályokat @Transactional annotációval láttam el, így az összes service metóduson keresztül történő művelet tranzakcióban folyik.

\subsection{AuthService}

Az AuthService feladata a felhasználók hitelesítése és azok jogosultságainak ellenőrzése. A felhasználók bejelentkezésekor a megdott felhasználónevet és jelszavat veti össze az adatbázisban tárolt felhasználónévvel és dekódolt jelszóval. A sikeres bejelentkezéskor az AuthService létrehoz egy Session objektumot, amely a felhasználónevet (ez minen felhasználónál egyedi), a session lejáratának időpontját, valamint a session id-t tartalmazza.  A csak bejelentkezett felhasználók számára elérhető végpontokban az AuthService metódusait használom arra, hogy az authorisation header-ben érkező session id alapján a felhasználó jogosult-e a végpontjoz tartozó művelet végrehajtására.

\subsection{EventService}

Az EventService-ben történnek az események kezelésével kapcsolatos műveletek, a naptár kezelésének kivételével. Kihívások az írása során:

Az események résztvevőinek jogosultsága van abból eseménysémát készíteni. Ekkor az EventService az esemény összes extra mezőjéből és azoknak a lehetséges értékeiből másolatot készít, és ezeket egyetlen kollekcióba gyűjti össze, majd JSON objektumként tárolom el az eseményséma (EventBlueprint) adatbázistáblában egy generált ID-vel és egy felhasználó által megadott kulcsszóval együtt. A séma felhasználható új események létrehozásakor, ekkor a JSON objektumot deszerializálom, az így elállt extra mezőket a lehetséges értékeivel elmentem az adatbázisba, majd azokat az új eseményhez rendelem.

\subsection{CalendarService}

A naptárak kezelése a CalendarService-en keresztül történik. Kihívások az írása során:

Az óraátállítások fennállásának ténye nehézségeket okozott, hiszen nem akartam lehetővé tenni, hogy a naptár felületen nem létező órák elérhetőek legyenek (ilyenek azok az órák amelyeket a tavaszi óraátállítás során kimaradnak), és azt sem, hogy kimaradjanak olyan órák amelyek a megszokott 24 órán kívül vannak (ezek az őszi óraátállítás során betoldott órák). Mivel az óraátállítások időzónánként eltérnek, a naptárak létrehozásakor paraméterben elvárom az időzóna megadását is. Ezután az időzóna segítségével Java ZonedDateTime objektumokat készítek a naptár kezdő és záró időpontjából, és a ZonedDateTime metódusainak segítségével generálom le a közöttük levő órákat. Ily módon a két időpont közötti összes óra benne lesz a naptárban, és egyetlen kimaradó órát sem fog tartalmazni.

Az legjobb időpontok megtalálása az egyik legfontosabb funkciója az alkalmazásnak. Ez a getBestTimeIntervals függvényen keresztül történik, amely négy paramétert vár: a naptárat, a résztvevők minimális számát, az időpontok minimális hosszát, és az elfogadott felhasználói véleménytípusok listáját. A függvény minden órához kiszámítja, hogy milyen hosszú az az ettől az órától kezdődő időszak, amely eleget tesz a paraméterekben megadott minimális hossznak és minimális résztvevőszámnak. Egy ilyen időszak esetében az számít részvevőnek, aki az az összes órára a paraméterben megadott véleménytípus egyikét választotta.

\subsection{MailService}

A felhasználóknak küldött rendszeren belüli üzenetek kezelését végzi. Kihívások az írása során:

A felhasználók üzeneteket kapnak az esemény lezárásáról, illetve az eseményt lezáró művelet esetleges visszavonásáról. Az események lezárásaról tájékoztató üzenet tartalmazza az eseményhez tartozó mező-érték párokat (ebbe az esemény kezdete és vége is bele tartozik, amennyiben naptárt is rendelt a szervező az eseményhez). Ezeket egyszerű szövegként nehéz értelmezni, ezért ezt formázott szövgként táblázatos formában terveztem megjeleníteni. Mivel a frontend oldalon nem terveztem módosításokat végrehajtani a backendről érkező levéleken, ezért a formázott üzenetet a backenden készítettem el, HTML formátumban. Mivel nem találtam az igényeimnek megfelelő Java könyvtárat, ezért készítettem egy HTMLWriter osztályt amelyben HTML elemeket előállító statikus metódusokat írtam. A formázott levelek frontend oldalon történő kezeléséről lejjebb értekezem.

\section{Controller réteg}

A controller osztályok szintén funkciók szerint különülnek el, ezért mindegyik tartalmaz hivatkozást a releváns service komponensérere, valamint az AuthService-re, amely a védett végpontok esetében végzi el a felhasználók autorizációját. Minden controller osztályhoz egy útvonal , a REST végpontokat megvalósító metódusaihoz pedig egy-egy alútvonal tartozik.

\subsection{Szerializálció, deszerializáció}

A model osztályaim példányait nem lehet egy-az-egyben JSON objektumokká alakítani. Az egyik probléma az osztályok közötti körkörös hivatkozások fennállása. Ezt tartalmazó kapcsolatok esetén úgy oldottam meg, hogy a tartalmazott osztályban Jackson @JsonIgnore annotációval tiltottam meg a tartalmazó osztály kapcsolódó példányának szerializálását, egyenrangú kapcsolatok esetén pedig azt teljesen ignoráltam a szerializáció során, a kapcsolat résztvevőit csak külön lekérések által tettem elérhetővé. Egyes osztályok esetén fennállhat az igény arra, hogy egyes tulajdonságait csak deszerializálni lehessen. Ilyen például a felhasználó osztály jelszó mezeje, hiszen, bár titkosítva van, ezt nem célszerű a frontend felé továbbítani, ellenben a deszerializálására szükség lehet a regisztráció során. Ezt úgy oldottam meg, hogy az adott tulajdonságot @JsonProperty(access = JsonProperty.Access.WRITE\_ONLY) annotációval láttam el.

\section{Kivételkezelés}

Külön kivételosztályokat hoztam létre az események, felhasználók, naptárak kezelése, illetve felhasználók autorizációja során keletkező kivételeknek. A speciálisabb kivételosztályokat ezekből az ősosztályokból öröklődtettem. A controllerek működése során keletkező, el nem kapott kivételek részére létrehoztam egy ControllerAdvice interfészt implementáló globális kivételkezelő osztályt. Ebben megadtam, hogy a keletkezett kivétel függvényében a visszaküldendő válasz milyen üzenetettel és státuszkóddal rendelkezzen.

\section{Tesztelés}

Minden service és controller komponenshez saját tesztosztályt hoztam létre, amelyeket az absztrakt EowaIntegrationTest osztályomból származtattam.  A tesztekhez egy properties fájlt is létrehoztam, ebben megadtam, hogy a tesztek ne a MariaDB adatbázisomat, hanem egy memórián belüli H2 adatbázist használjanak. A service-ekhez tartozó tesztek írása során törekedtem a teljes tesztlefedettségre, a controller-ekhez tartozó tesztek esetében csak annyi tesztet írtam, amennyiből megállapítható, hogy a szerializálás, a deszerializálás, és a kivételek kezelése megfelelő módon történik. A controllereket MockMvc használatával teszteltem. A teszteseteket az "3 A" (Arrange, Act, Assert) szabály szerint készítettem el, azaz először elkészítettem a tesztesethez szükséges objektumokat, ezután elvégetzem a vizsgálandó műveletet, majd ellenőrzöm, hogy az elvárt eredményt sikerült-e elérni.

\section{Védelem}

	\subsection{CORS}

A Spring Security-ben lehetőség van saját CORS, azaz a Cross-origin Resource Sharing configuráció megadására. Ebben megadható, hogy a szerver mely domainekről fogad kéréseket, és azt is, hogy azok esetében mely kérés metódusokat és header-öket engedélyezi. A konfigurációmban kizárólag azt az útvonalat engedélyeztem, amelyen a frontend alkalmazásom érhető el, arról viszont az összes metódust és header-t elfogadtattam.


	\subsection{CSRF elleni védekezés}

A CSRF a Cross Site Request Forgery rövidítése. A CSRF lényege, hogy az egyik oldalra bejelentkezett felhasználó egy másik oldalról, tudta nélül küld kérést a szerver felé. Ennek nagy a kockázata session cookie-k használata esetén, mivel a böngészősütikhez más oldalak is hozzáférnek, ezért a session id-t nem sütikben, hanem az authorisation headerben várják az alkalmazás REST végpontjai, a frontend oldalon pedig a localstorage-ben tárolom azt, így más oldal nem férhet hozzá.

\chapter{A frontend fejlesztése}

\section{Model réteg}

Az egyes backend oldali model osztályoknak megfelelő TypeScript interfészeket írtam, hogy megfelelően szerializálhassam és deszerializálhassam őket. Ezek nem tekinthetőek viewmodel osztályoknak, hiszen nem csupán a megjelenítés a céljuk, pédául az id tulajdonságokat is tartalmazzák.

\section{Service réteg}

Az ApiService-ben írtam meg a fetch API-t használva a backend felé HTTP kéréseket küldő metódusokat, valamint a kéréshez tartozó header-öket is itt állítom össze (utóbbiakból a legfontosabb az authorisation header, amelybe az aktuális felhasználó session információi kerülnek). A többi service osztályba pedig az ApiService-t injektáltam, így az összes kérés rajta keresztül zajlik.  Mivel az üzleti logika kizátólag a backend oldalon található, az összes service metódus egyetlen, az API felé küldött kérésből és a válasz feldolgozásából áll.

\section{Pipe-ok}

Az angular Pipe-ok segítségével egyszerűen lehet a template fájlokban JS objektumokat megjeleníthető formába alakítani. Én több ilyet is létrehoztam, például egy pipe segítségével jelenítem meg a pontos dátumot a naptárfelületen. Itt a pipe transform metódusa a naptár objektumot és az óra sorszámát számát kapja meg, majd ezek segítségével adja vissza a pontos dátumot leíró sztringet, vagy az "end of the event" szöveget.

\section{Navigáció az oldalon}

A komponensek közötti navigáció megvalósításához azokba az Angular Router service-t injektáltam, amely az app.routes fájlban tárolt útvonalak és az azokhoz társított komponensek és guard függvények segíségével végzi el a szükséges műveleteket. Egyes esetekben query paramétereket is használtam, ezeket a komponensekbe injektált ActivatedRoute segítségével kértem le.

\section{Fő komponensek}

	\subsection{Dialógusablak}

Saját dialógusablak komponenst hoztam létre, amely alkalmas formázott és formázatlan üzenetek megjelenítésére is. A formázot szöveg megjelenítését az események lezárultát tudtul adó üzenetek esetén használom, hiszen ezek táblázatokat is tartalmaznak.  A HTML formában megadott üzeneteket egyirányú kötéssel a dialógus komponens InnerHTML tulajdonságához rendeltem, és a stílust is a frontend oldalról adtam meg, hogy azt ne kelljen redundáns módon az adatbázisban tárolni minden üzenet rekordban.

	\subsection{Naptárfelület}

A naptárfelület több standalone komponensből tevődik össze. A ViewCalendarComponent-en belül az egyes napokhoz, és azokon belül az órákhoz külön komponens tartozik. Mivel az EventService-t csak a ViewCalendarComponent-ben injektáltam, a DayComponent-ben és az HourComponent-ben Input és Output direktívákat alkalmaztam, hogy az adatok mindkét irányban megfelelő módon áramoljanak.

\section{Űrlapok reaktív módon történő kezelése}

Az egyszerű, egyetlen input mezőből álló és frontend oldali validációt nem igénylő űrlapokat kétirányú adatkötéssel dolgoztam fel, az összetettebbeket azonban a ReactiveFormsModule segítségével alkottam meg. Az űrlapokhoz FormGroup-okba gyűjtött FormControl objektumokat rendeltem, amelyekhez validációs függvényeket is írtam, ahol a ReactiveFormsModule előre megírt függvényei nem voltak elegendőek.

\section{UX, UI}

	\subsection{Guard-ok}

Az Angular-ban a guard-ok olyan függvények, amelyek segítségével eldönthető, hogy a felhasználó navigálhat-e egy adott al-útvonalra az oldalon belül. Használatuknak jellegzetes esete gaurd-ot rendelni azokhoz az útvonalakhoz, amelyeket csak bejelentkezett felhasználók használhatnak. Ez UX funkció, hiszen az ilyen útvonalakhoz rendelt komponensekből olyan funkciók is elérhetőek, amelyekre a választ a szerver vissza fogja utasítani, ha nincs bejelentkezve a felhasználó, ennek előfordulása egy guard használatával elkerülhető. Én is írtam egy ilyen függvényt, amelybe a UserService-t injektáltam, hogy annak getCurrentUser metódusa segítségével eldönthesse, jogosult-e a felhasználó az útvonal aktiválására.

	\subsection{Folytonos színátmenet style binding segítségével}

A naptárfelületen színek segítségével jelenítem meg, hogy az egyes órák mennyire népszerűek a résztvevők körében Az óra komponensben létrehoztam egy függvényt, amely annak népszerűségét a felhasználói vélemények alapján egy nulla és egy közötti szám formájában adja vissza. Ennek függvényében akartam a komponensek háttérszínét a pirostól a zöld színig terjedő skálán elhelyezni, ezért egy új tulajdonságot vezettem be a komponensben, amelyben a háttérszínt RGB formátumban tároltam. A kék összetevőt nullára állítottam, a zöld összetevőt a népszerűséggel egyensen arányos módon, a piros összetevőt pedig a fordítottan arányos módon állítottam be. A tulajdonságot pedig a template fájlban stíluskötéssel állítottam be háttérszínnek.

	\subsection{Canvas Angularban}

Az óra komponensekre kattintva megjelenik egy dialógusablak, amely információkkal szolgál az adott órához tartozó véleményekről. Itt egy egyszerű diagramot is elhelyeztem, hogy az információk könnyebben feldolgozhatóak legyenek. Ezt egy beépített Canvas segítségével alkottam meg. A komponens fájlban egy, ViewChild dekorátorral konfigurált HTMLCanvasElement típusú ElementRef segítségével tudok a template fájlban megdott canvas-re hivatkozni.

\section{Aszinkron programozási megoldások}

Az aszinkron műveletekhez az RxJs könyvtár eszköztárát használtam. Hogy ez mindenhol egységes legyen, a Promise-okat visszaadó függvények eredményeit RxJs Observable objektumokká alakítottam, ilyen az összes, HTTP kérést tartalmazó service függvényem, hiszen ezek a fetch API-t használják. Az RxJs Subject típusát is használtam arra az esetre, amikor szülő komponensben keletkező eseményre a gyerek komponensnek kell reagálnia.
 
\chapter{Továbbfejlesztési lehetőségek}

Az egyik lehetőség a továbbfejlesztésre a JWT (Jason Web Token) használata az autorizációs folyamatokra. A jelenlegi megoldásom, bár a localstorage-ben tárolt json objektum segítségével zajlik, nem nevezhető jwt-nek, mivel az autorizáció a benne tárolt session id alapján történik, amitnek hitelességét a backend oldalon kell lellenőrizni, és nem egy digitális aláírás segítségével.

Az alkalmazás reszponzivitása javítható, jelen állapotban például mobilról nehezen használható. Emellett a felhasználói felület akadálymentesítésével is érdemes lehet foglalkozni.

Az értesítések nagyobb valószínűséggel juthatnak el a felhasználókhoz, ha megvalósítanám, hogy az értesítéseket a megadott emal-címükre is megkapják a felhasználók.

\chapter{Hasonló alkalmazások}

Bár én a projekt megkezdése előtt nem hallottam róla, az alkalmazás írása közben többen felhívták rá a figyelmemet, hogy hasonló célra készült alkalmazások már léteznek, ezért ezeknek utánanéztem, hogy összehasonlíthassam őket a saját megoldásommal. A legtöbbet a Doodle alkalmazást említették, emellett a ClickUp, valamint a WhenAvailable merült még fel. Az alábbiakban az alkalmazásomat a Doodle ingyenes változatával vetem össze.

\section{Doodle}

A Doodle-ban több, az enyémhez hasonló megoldás is van, például az eseményhez csatlakozás egy megosztott linken keresztül történik, valamint itt van egy esemény véglegesítő művelet, amelyről a résztvevők értesítést kapnak.  További hasonlóság, hogy lehetőség van az egyszerű "megfelel" mellett az "amennyiben nincs más lehetőség" válasz is elérhető. Különbség, hogy ebben csak előre megadott, konkrét időpontokra lehet jelentkezni, valamint az is, hogy az esemény más tulajdonságait (pédául a helyszínt) a szervező nem teheti szavazás tárgyává. Különbség az is, hogy a Doodle-ben regisztráció nem szükséges, az értesítéseket az eseményhez társított email-címükre kapják a felhasználók. A doodle-ben a véglegesítési művelet nem vonható vissza, és mivel extra mezőket nem lehet az eseményekhez társítani, következtetésképp azokból eseménysémákat sincs lehetőség létrehozni.

\chapter{Összefoglalás az eredményekről}

\chapter{Összefoglalás}

(új oldalon kezdve)

A dolgozat eredményeinek összefoglalása, következtetések levonása.

Az összefoglalásban egyértelműen jelezve legyen a hallgató saját szerepe/eredményei.

\newpage

\bibliographystyle{unsrt}
\bibliography{mate_koncz} 

\newpage
{\Huge \bf Köszönetnyilvánítás}

 \addcontentsline{toc}{chapter}{Köszönetnyilvánítás}

\vspace{2 cm}

(nem kötelező elem), (új oldalon kezdve) 

Ebben a fejezetben lehet köszönetet mondani mindazoknak, akik segítették a dolgozat elkészülését. Itt lehet megemlíteni továbbá a munkát támogató pályázatokat, ösztöndíjakat, stb.

\newpage
{\Huge \bf Nyilatkozat}

 \addcontentsline{toc}{chapter}{Nyilatkozat}

\vspace{2 cm}

{\it A szöveg kötött, kérjük ezt használni!}

Alulírott, Végzős Edömér, xxxx szakos hallgató, kijelentem, hogy a szakdolgozatban ismertetettek saját munkám eredményei, és minden felhasznált, nem saját munkából származó eredmény esetén hivatkozással jelöltem annak forrását. 


\begin{flushleft}
\vspace*{1cm}
Szeged, \today
\end{flushleft}

\begin{flushright}
 \vspace*{1cm}
 \makebox[7cm]{\rule{6cm}{.4pt}}\\
 \makebox[7cm]{\emph{Végzős Edömér}}
\end{flushright}

\pagebreak

\newpage
{\Huge \bf Mellékletek}

 \addcontentsline{toc}{chapter}{Mellékletek}

\vspace{2 cm}

(nem kötelező elem, a dolgozat oldalszámaiba nem tartozik bele.), (új oldalon kezdve)

Ebben a fejezetben lehet elhelyezni a nagyobb táblázatokat, ábrákat, adathalmazokat.


\end{document}



